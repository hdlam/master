\chapter{Evaluation and Conclusion}
\label{cha:evaluationAndConclusion}



%{\it Lorem ipsum dolor sit amet, consectetur adipiscing elit. Nam consequat pulvinar hendrerit. Praesent sit amet elementum ipsum. Praesent id suscipit est. Maecenas gravida pretium magna non interdum. Donec augue felis, rhoncus quis laoreet sed, gravida nec nisi. Fusce iaculis fermentum elit in suscipit. }

\section{Evaluation}
\label{sec:Evaluation}
All of the graphs shown in section \ref{sec:results} shows that the robots are able to flock together if given enough time. A cohesion distance of 800 px makes the robots move towards one another, making them flock together. There is a similarity between the result data generated from the physical experiment with the robots and the experiments ran on a simulator.

\textbf{Research question 1:} {\it Can a centralized computer aid a swarm of robots that do not have enough sensors to do the task it was assigned for?}

Each robot were equipped with eight distance sensors and a bluetooth module for communication. Other types of sensors can be equipped on the robots, but 
Using only the distance sensors are only able to detect other objects around the robots, but the robot can not use the distance sensors to distinguish what kind of object it is sensing. The structure of the robots makes it hard for the other robots to detect it due to the hollowness of the robot. 
By using a camera and a centralized computer which the robots can communicate with, the robots gains information about its whereabouts along with its angle and all the other robot's positions. 

\textbf{Research question 2:} {\it Will the Boids algorithm make the robots flock together?}

The robots gets the information about its location from the centralized computer, and the whereabouts of the other robots. Using this information the robot will calculate the Boids behaviors vector and move in the direction of the final vector.

Looking at the graphs from section \ref{sec:results}, we can see that the robots are moving closer to each other, and they stay together with approximately 35 cm between each robot. In scenario 2, the robots do move on each side of the obstacle due to their placement at the start. If a robot faces an obstacle directly in front of it, they will "flip a coin" whether they turn left or right, thus a randomly direction will be chosen around the obstacle if the robot is moving directly toward an obstacle.
However, the robots do not move together towards the obstacle, like the Boids in the simulator do. The most upper left robot seen in \ref{fig:scenario2} is trapped between the corner and the two robots around it. Having no way to move out at the start and thinking that the other robots around it are obstacles, makes it turn around back to the wall. This delays it from moving towards the real obstacle, because it has to turn all the way back again.
The robots avoids crashing into other objects at almost all cost, it is its highest priority. So when the robots are clumped together, they might spin a lot on the same spot.

%TODO

%When evaluating your results, avoid drawing grand conclusions, beyond that which your results can in fact support. Further, although you may have designed your experiments to answer certain questions, the results may raise other questions in the eyes of the reader. It is important that you study the graphs/tables to look for unusual features/entries and discuss these aswell as discussing the main findings in the results. 




\section{Contributions}
\label{sec:Contributions}
%contribution to AI, swarm, chirp


%What are the main contributions made to the field and how significant are these contribution.



\section{Future Work}
\label{sec:futureWork}

%TODO write intro to this chapter here
This section will look further upon what research question is still unanswered and what can be improved.\\\\
\textbf{More robots}

The current implementation of the system only supports four simultaneous robots at the same time. A new robot can not be added to the system without uploading new code to each robot, nor can a robot disappear from the system. The camera tracking software loads a configuration file at startup that specifies how many robots will be tracked, this value has to be manually changed by the user and reloaded.
The watcher software needs to know which robot ID belongs to which bluetooth serial port. If one of the robots' bluetooth connection disconnects or times out, everything has to stop and reconnected again for the system to work.

The main idea behind swarm robots is that it should be able to run continuously without having to reboot or stop the swarm to add a new robot or remove one. If one of the entity in a swarm is defected, injured or not working, the other ones should still be operable.

The current implementation of this project relies on the robot needing to know how many other robots there will be in the sandbox at the same time, because it will need to know how much data it is going to store before processing it. This has to be changed manually in the code and then uploaded to the robot again.
A working flocking flock should have more than four robots running simultaneously, and we should be able to add a new robot without re-uploading the code to the robots.

In simulation, it is easy to add new Boids of different type or family as seen in figure \ref{fig:simulatordistances} by the different colors on the Boids. With only four robots running at the same time, there is no point in trying to implement different types of Boids families so sub groups of Boids flocks would emerge. There reason is that it will be hard to see if there is any family groups with only four robots. With more robots, it would be easier and more obvious to see groups of families emerge. \\\\
\textbf{More fluid movement}

The current implementation does not punish the robot for standing still nor for only rotating on the same spot. The only escape from continuously spinning on the same spot is a spin counter, which forces the robot to move forward if the robot has been stuck on a spot for five iterations and if there is nothing in front of it. 

Each robot also stops when it turns around and whenever they need to change direction. When birds fly together as a flock they do not stop when they need to turn. The robots should be able to move and rotate without stopping.

In the simulator, the Boids held its formation whenever they flocked until they had to change their direction because they were too near an obstacle. The robots do stay near each other to a certain degree. Whenever the robots flock, the distances between each robot varies, they do not have a consistent spacing which the Boids on the simulator do have. The reason this happen might be because the robots are not fully synchronous like the simulator Boids are, that is each Boids calculates their new vector direction every frame at the same time, while the robots do not calculate their new direction at the same time as the other robots. 

The robots still bumps into the obstacle and they still bump into each other. A rewrite of the executive layer should be able to fix this problem. Instead of checking the distances to objects in front of it before it is about to move in a direction, the robots should check and measure the distances continuously. Checking the distance all the time might affect movement of the robot. But if the sensors could be more precise, then the disturbances and noises would not be a problem for the movement of the robot.

In scenario 2, the robots do move around the obstacle, but this does not happen at the same time. The three robots stays a while in the upper left corner before moving, and they do not move as a unit to the other side of the sandbox. Some of the runs, they were moving one robot at a time. 


%[From template] %TODO remove
%Consider where you would like to extend this work. These extensions might either be continuing the ongoing direction or taking a side direction that became obvious during the work. Further, possible solutions to limitations in the work conducted, highlighted in ~\ref{sec:Discussion} may be presented. 



