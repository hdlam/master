\section*{Abstract}
\label{sec:abstract}

This thesis gives a brief overview to the field of swarm robotics, and investigates how robots are able to flock together using the Boids algorithm. Four robots will be used for the experiments. Swarm robotics is an emergent field in artificial intelligence. The advantage of flocking robots is that the overall system becomes more stable, more robust because each robot contributes to the overall robustness of the swarm. Boids algorithm is an algorithm meant for computer animation or computer aided design and has been used in various movies.

In this thesis a Boids simulator will be implemented to get a hands-on experience with the behavior of the Boids. Then the Boids algorithm will be implemented on four differential wheeled robots which has only distance sensors equipped, and a bluetooth module. The robots are not able to distinguish obstacles from other robots using their distance sensors only, therefore a centralized computer with an attached camera to provide data to the robots that they do not have access to.
The robots used in this thesis is a differential wheeled robot called the ChIRP robot, which is an open source robot made by the CRAB lab at the Artificial Intelligence section of the Department of Computer and Information Science, Norwegian University of Science and Technology.



%This paper provides a template for writing AI project rapports for either the AI specialization project; masters "datateknikk" or masters "informatikk". The use of the template is recommended and is written in English as we encourage students to submit their project and masters theses in English. 
%The template does not form a compulsory style that you are obliged to use. However, the format and contents are a result of a joint AI group initiative thus providing a common starting point for all AI students. For a given project tuning of the template may still be required. Such tuning might involve moving a chapter to a section or vice versa due to the nature of the project. 
%
%The abstract is your sales pitch which encourages people to read your work but unlike sales it should be realistic with respect to the contributions of the work. It should include:
%\begin{itemize}
%\item the field of research *
%\item a brief motivation for the work
%\item what the research topic is and
%\item the research approach(es) applied. 
%\item contributions
%\end{itemize}
%
%The abstract length should be roughly half a page of text --- without lists, tables or figures.  

\clearpage

\section*{Preface}



\vspace{1cm}

%The preface includes the facts - what type of project, where it is conducted, who supervised and any acknowledgements you wish to give. 
First I want to thank my supervisor Professor Keith Downing, at the Department of Computer and Information Science, Norwegian University of Science and Technology, for the guidance through this project. This project would not be possible without him.

I would like to thank Christian Skjetne one of the creator of the ChIRP robot for helping with the technical difficulties that occurred when using the robots on both the bluetooth communication and the setup of the Arduino libraries. A lot of the library code that the robot's motor and sensors needs to function are written by Christian. The camera tracking software was written by Erik Samuelson, which was an essential part of this project.

I would specially mention the team that the Java gaming library named Slick2D, this gaming library makes it easier to render shapes, and text on screen.
All references are created using citethisforme.com and the software named Mendeley desktop, they have saved me for a lot of work by automating the process of referencing articles.
