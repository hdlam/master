\section{Goals and Research Questions}
\label{sec:Goals and Research Questions}


A masters is a research project and thus there needs to be a question(s) that need answered. Such questions are often a very important part of the results that come out of the specialisation project. For those following the one year masters project, it is desirable to create such questions as early as possible as   The formation of such questions provide both an important driving force for the masters project and provide clarity as to the goals sought. However, one will expect to refine the questions and thus the final path of the masters as work progresses. However any refinements should be conducted with care so as to avoid that the original aims, and previous work are not lost.  
It is always good to have one (or max 2) key questions and perhaps some sub questions. 

\begin{description}
\item[Goal] {\it Lorem ipsum dolor sit amet, consectetur adipiscing elit.}
\end{description}

Your goal/objective should be described in a single sentence. In the text under you can expand on this sentence to clarify what is meant by the short goal description. 
The goal of your work is what you are trying to achieve. This can either be the goal of your actual project or can be a broader goal that you have taken steps towards achieving. Such steps should be expressed in the research questions. 
Note that the goal is seldom to build a system. A system is built to to enable experiments to be conducted. The research question/goal would be the goal that the system is implemented to meet.  


\begin{description}
\item[Research question 1:] {\it Can a centralized computer aid a swarm of robots that do not have enough sensors to do the task it was assigned for?}
\end{description}



\begin{description}
\item[Research question 2:] {\it Lorem ipsum dolor sit amet, consectetur adipiscing elit.}
\end{description}

{\it Lorem ipsum dolor sit amet, consectetur adipiscing elit. Nam consequat pulvinar hendrerit. Praesent sit amet elementum ipsum. Praesent id suscipit est. Maecenas gravida pretium magna non }

Research method

To be able to address the research questions, a simulator was created to run the Boids algorithm in simulation. The purpose of the simulator was to be able to see how each Boids are supposed to behave, and the code from the simulator could easily be ported to the real robots afterwards. The simulator also serves as a starting point for the real robots, for debugging purpose.

%TODO write something about how the robots were used.
%graphing of mean and sd