\chapter{Introduction}
\label{cha:Introduction}

This master thesis researches the flocking behavior of the Boids algorithm by implementing it on four differential wheeled robot.

This chapter is the introduction of the thesis, section \ref{sec:BackgroundAndMotivation} contains background and motivation for this thesis, section \ref{sec:Goals and Research Questions} contains the research question that this thesis is based on. A brief overview of the contribution this thesis have made to this field is presented in section \ref{sec:IntroContributions}. An overview of the content in this thesis can be found in section \ref{sec:thesisStructure}.

%All chapters should begin with an introduction before any sections begin. Further, each sections begins with an introduction before  subsections begin. Chapters with just one section or sections with just one sub-section, should be avoided. Think carefully about chapter and section titles as each title stand alone in the table of contents (without associated text) and should convey meaning for the contents of the chapter or section. 

%In all chapters and sections it is important to write clearly and concisely. Avoid repetitions and if needed, refer back to the original discussion or presentation. Each new section, subsection or paragraph should provide the reader with new information and be written in your own words. Avoid direct quotes. If you use direct quotes, unless the quote itself is very significant, you are conveying to the reader that you are unable to express this discussion or fact yourself. Such direct quotes also break the flow of the language (yours to someone else's).   

\section{Background and Motivation}
\label{sec:BackgroundAndMotivation}

Swarm robotics is an emergent field in artificial intelligence. Swarms are inspired by nature, especially animals that works together. There are many reasons for animals to flock together. Fishes flocks together to increase their survivability from being eaten by preys. Ants flock together to find food, and birds flock together to increase survivability and to minimize air resistance.

Usually flocks do not have a leader, they are able to operate on their own. Each individual in the flock have simple behaviors, but the collective behavior can be quite complex. Swarm robots are decentralized units, meaning that they are not controlled by one centralized unit. Each of them interacts with each other through local interactions, and exchanging information to achieve their goal. Losing one or a few of the robot should not affect the swarm as a whole.

Robots are becoming more and more relevant in our lives, already nowadays there are robots helping people mow the lawn, robots that are vacuuming houses etc. Traditional robots are being used for these types of tasks, that is bigger expensive robot. Having smaller cheaper robots might be advantageous in some situations, they are smaller and cheaper. Due to their size, they might be able to reach places that the bigger traditional robots might not be able to reach. Because the swarming robots are usually very cheap, they can easily be replaced when they are malfunctioning. The idea is that the small swarm robots will be able to do the same task as the bigger traditional robot. Robot swarm moving in a flock can for example be used to explore new areas. Moving in a swarm makes the exploration more susceptible to failures. Swarm robots can work together to move over rough environments like slippery roads, and they are able to move trough narrow passages because they are small.

The CRAB lab at the AI department at NTNU have created a robot called the ChIRP robot. These robots are open source and can easily be mass produced. The idea behind these robots was to research swarm robotics.
These robots can easily be expanded with other types of sensors, but only distance sensors and light detecting sensors are available at the moment. These robots are still in an early phase of development and has yet to be used to their full potential. 


%To make the robots flock together the Boids algorithm will be used. Due to the lack of sensors, a centralized unit 


%Having a template to work from provides a starting point. However, for a given project, a slight variation in the template may be required due to the nature of the given project. Further, the order in which the various chapters and sections will be written will also vary from project to project but will seldom start at the abstract and sequentially follow the chapters of the report. One critical reason for this, is that you need to start writing as early as possible and you will begin to write up where you are currently focusing. However, do not leave the abstract until the end. The abstract is the first thing anyone reads of an article or thesis --- after the title; and thus it is important that it is very well written. Abstracts are hard to write so create revisions throughout the course of your project as your project progresses.  

%This introduction to background and motivation should state where this project is situated in the field and what the key driving forces motivating this research are. However, keep this section brief as it is still part of the introduction. The motivation will be further extended in chapter~\ref{T-B}, presenting your complete state-of-the-art. 

%Note that this template uses italics to highlight where latin wording is inserted to represent text and the text of the template that we wish to draw your attention to. The italics themself are not an indication that such sections should use italics.  


\section{Goals and Research Questions}
\label{sec:Goals and Research Questions}

%
%A masters is a research project and thus there needs to be a question(s) that need answered. Such questions are often a very important part of the results that come out of the specialisation project. For those following the one year masters project, it is desirable to create such questions as early as possible as   The formation of such questions provide both an important driving force for the masters project and provide clarity as to the goals sought. However, one will expect to refine the questions and thus the final path of the masters as work progresses. However any refinements should be conducted with care so as to avoid that the original aims, and previous work are not lost.  
%It is always good to have one (or max 2) key questions and perhaps some sub questions. 
%
%\begin{description}
%\item[Goal] {\it Lorem ipsum dolor sit amet, consectetur adipiscing elit.}
%\end{description}
%
%Your goal/objective should be described in a single sentence. In the text under you can expand on this sentence to clarify what is meant by the short goal description. 
%The goal of your work is what you are trying to achieve. This can either be the goal of your actual project or can be a broader goal that you have taken steps towards achieving. Such steps should be expressed in the research questions. 
%Note that the goal is seldom to build a system. A system is built to to enable experiments to be conducted. The research question/goal would be the goal that the system is implemented to meet.  



\textbf{Research question 1:} {\it Can a centralized computer aid a swarm of robots that do not have enough sensors to do the task it was assigned for?}

Swarm robotics are mostly focused on many small cheap robots working together in a swarm to achieve their goal. The swarm robots usually can be changed while the system is up and running without affecting the flock.
Due to the cheapness of the robots, they might have cheaper equipments and sensors or they might lack the necessary equipment for robots to be able to its task. 
The robots used in this thesis only have distance sensors, but can add more sensors and modules if needed. However there are not other types of sensors available, therefore the robots will communicate with a centralized computer which might be able to aid the robot gain information that it currently lack of its surrounding.
Communication between the robots are limited, because the communication between them has to go trough the centralized computer.


\textbf{Research question 2:} {\it Will the Boids algorithm make the robots flock together?}

The Boids algorithm uses vectors for each behavior to steer the entities around. The minimal number of behaviors for a typical Boid is three behaviors. One of them ensures that the entities flock together, the other one makes sure that they move together when they have flocked. The last behavior ensures that they do not collide with each other. The robots will gain information about their whereabouts from the centralized computer, and the location of the other other robots. Using this information the robots will process the information using the Boids algorithm to flock together.


\section{Research Method}
\label{sec:researchMethod}
%What methodology will you apply to address the goals: theoretic/analytic, model/abstraction or design/experiment? This section will describe the research methodology applied and the reason for this choice of research methodology.  
To be able to address the research questions, a simulator was created to run the Boids algorithm in software. The purpose of the simulator was to be able to see how each Boids are supposed to behave, parts of the code from the simulator can be ported to the real robots afterwards. The simulator also serves as a starting point for the real robots, for debugging purpose.
The Boids algorithm will be used to make the robots flock and move together. Distance sensors will be used to avoid obstacles.
Three scenarios will be used to test if the Boids behaves the way they are supposed to do. The scenarios will be designed to test that the robots are flocking together, and avoiding obstacle by moving around it.

To evaluate the behaviors of the robots, the mean and standard deviation of the robot's angle, velocity and distance between the robot will be graphed. 

%TODO write something about how the robots were used.
\section{Contributions}
\label{sec:IntroContributions}
TODO
%The main description of the contributions will come in chapter~\ref{cont} after the results are presented. This section just provides a brief summary of the main contributions of the work. This section can also be left out, leaving all discussions in chapter~\ref{cont}.

%The format of this section will generally follow the following format:
%{\it
%Donec non turpis nec neque egestas faucibus nec id neque. Etiam consectetur, odio vitae gravida tempus, diam velit sagittis turpis, a molestie ligula tellus at nunc. Nam convallis consequat vestibulum. Proin dolor neque, dapibus a pellentesque a, commodo a nibh.}

%\begin{enumerate}
%\item {\it Lorem ipsum dolor sit amet, consectetur adipiscing elit.}
%\item {\it Lorem ipsum dolor sit amet, consectetur adipiscing elit.}
%\item {\it Lorem ipsum dolor sit amet, consectetur adipiscing elit.}
%\end{enumerate}

\section{Thesis Structure}
\label{sec:thesisStructure}
This thesis is divided into five chapters. Chapter \ref{cha:TheoryAndBackground} contains research and background information from the field of swarm robotics and the bird flocking algorithm, namely the Boids algorithm. A brief overview of the most common robot control architecture is explained in the same chapter.

In chapter \ref{cha:architectureAndModel} the architecture and the model of the system is explained. The chapter goes in detail on how the system is set up and how the computers are connected.
Chapter \ref{cha:ResearchAndResults} contains the results, graphs and data gathered from running the experiments and a discussion about them. The last chapter \ref{cha:evaluationAndConclusion} contains the evaluation of the results and a further work section where the details on how to improve the system is explained.

%An appendix on the end of this thesis contains information which might come 
%This section provides the reader with an overview of what is coming in the next chapters. You want to say more than what is explicit in the chapter name, if possible, but still keep the description short and to the point. 