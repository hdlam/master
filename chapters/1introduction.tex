\section{Introduction}
\label{sec:intro}
This project report is a report for the specialization project named\\ \textbf{INTELLIGENT ROBOTICS: Adaptive Robots}. 
\textit{Robots are commonplace in society; they perform all sorts of jobs. However, most are hard-wired to perform a few fixed tasks and have little or no ability to adapt to changing or unforseen circumstances. Artificial Intelligence (AI) enters the picture when robots must be capable of autonomous behavior in unpredictable environments. In this project, the student must utilize AI machine-learning techniques to produce a robot with adaptive capabilities. Possible techniques include reinforcement learning (RL), artificial neural networks (ANNs), evolutionary algorithms (EAs), or combinations of these, such as evolving neural networks.}

\textit{The current options for programmable robots are a) the CHiRP robots being designed in the CRAB Lab of the AI Group, b) larger robots being developed in the Cybernetics Department.}

The decision to use the ChIRP robot instead of working with the Cybernetics Department was due to availability of the robot. It would be easier to have all the equipment available.

The project description that fits this project the most is: \\\textbf{INTELLIGENT ROBOTICS: Swarm Intelligence}
\textit{Students will work with groups of simple agents (e.g. simulated or physical robots) to investigate the emergence of global structure from local interactions. No global controller will dictate the agents' actions.}\\

This report contains background information about swarming birds, and how three simple behaviors are applied to animated particles to make the flock like real birds found in nature.
Section \ref{sec:background} contains background information, it explains shortly what swarming is, how swarms can be used to simulate animals found in nature, more precise how to simulate bird flocking using the Boid algorithms.

Section \ref{sec:chirp} gives some background information on the robot platform used in this project and the upcoming master project. 

Section \ref{sec:prepro} contains an overview of the system used to demonstrate platooning at Statens Vegvesenet's Teknologidagene, how the system was implemented. Issues with implementation of the system is addressed, and fixes are suggested to overcome these issues.

Section \ref{sec:master} provides information about the upcoming master project in spring 2015. This is just an overview of what to expect. Issues that might occur are addressed, but some unexpected issues might occur as well.
