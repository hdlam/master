\section{Camera tracking configuration file}
\label{app:opencvcfg}
chirpObserverConfig.cfg
\begin{lstlisting}
# [comPort]
# the id of the comm port to use. 
# 0 corresponds to /dev/ttyS0 
# 1 corresponds to /dev/ttyS1
# 2 corresponds to /dev/ttyS2
# 3 corresponds to /dev/ttyS3
# 4 corresponds to /dev/ttyS4
# 5 corresponds to /dev/ttyS5
# 6 corresponds to /dev/ttyS6
# 7 corresponds to /dev/ttyS7
# 8 corresponds to /dev/ttyS8
# 9 corresponds to /dev/ttyS9
# 10 corresponds to /dev/ttyS10
# 11 corresponds to /dev/ttyS11
# 12 corresponds to /dev/ttyS12
# 13 corresponds to /dev/ttyS13
# 14 corresponds to /dev/ttyS14
# 15 corresponds to /dev/ttyS15
# 16 corresponds to /dev/ttyUSB0
# 17 corresponds to /dev/ttyUSB1
# 18 corresponds to /dev/ttyUSB2
# 19 corresponds to /dev/ttyUSB3
# 20 corresponds to /dev/ttyUSB4
# 21 corresponds to /dev/ttyUSB5
# 22 corresponds to /dev/ttyAMA0
# 23 corresponds to /dev/ttyAMA1
# 24 corresponds to /dev/ttyACM0
# 25 corresponds to /dev/ttyACM1
# 26 corresponds to /dev/rfcomm0
# 27 corresponds to /dev/rfcomm1
# 28 corresponds to /dev/ircomm0
# 29 corresponds to /dev/ircomm1
# id = 27

[camera]
# the id of the camera device to use, numbered from 0 and upwards
# deviceId = 1
deviceId = 0
# the width and height of the captured image, denoted in pixels
width = 720	
height = 500
# the maximum number of objects to track. 
#All objects are ignored if the number of objects
# discovered exceeds this number
maxNumObjects = 1000
#the minimum object area to recognize (in pixels). 
#Any object smaller than this is ignored.
minObjectArea = 10
# upper limit of the size of leds compared to min(frame width, frame height)
ledSize = 0.01

# settings governing the colors in the captured frame
# the settings are potentially specific to each camera
# linux application "guvcview" provides details
# the settings appear to be sticky, so an application like "guvcview"
# may be necessary to revert to default state
#
# if the value is less than 0, the application won't change
# the camera's setting for that property
# the brigtness of the image
brightness = 0.1
# the contrast of the image (0 - 10 with default of 5)
contrast = 5
# the saturation of the image (0 - 200 with default of 83)
saturation = 30

# each color is defined by six boundries,
#and is only recognized if the HSV value is within the bounds
[red]
hmin = 120
hmax = 220
smin = 100
smax = 256
vmin = 130
vmax = 256

#hmin = 0
#hmax = 45
#smin= 90
#smax = 256
#vmin = 170
#vmax = 256

[blue]
hmin = 100
hmax = 120
smin = 250
smax = 256
vmin = 250
vmax = 256

[green]
hmin = 31
hmax = 90
smin = 100
smax = 256
vmin = 135
vmax = 256

#hmin = 21
#hmax = 86
#smin= 50
#smax = 256
#vmin = 205
#vmax = 256

# the size of the virtual feed depicting the 
#scene as interpreted by the tracker
[virtualFeed]
width = 500
height = 500
 
# settings for the UDP network sending 
#information about tracked robots
[network]
enabled = true 
address = localhost
port = 52346

[general]
# whether to calibrate colors, this allows one to 
#find values for the color definitions
calibrationEnabled = false
# the time the system waits after a frame 
#has been processed before it fetches a new one
# if this number is set too low, 
#the camera won't be able to finish capturing a frame
frameDelay = 10

[tracking]
# number of robots to track
numRobots = 4
# robots are removed if this number of steps pass 
#without a pinpointing of the robot's position
evictionLimit = 6
# the maximum speed (distance / frame) of robots in any 
#direction compared to the width and height of the bounding box
 robotSpeed = 0.01
# robotSpeed = 0.065
# the maximum rotational speed (radians / frame) of
# the robot compared to a full circle
robotRotationalSpeed = 0.05
# the diameter of the robots compared to the bounding box
robotDiameter = 0.06
#0.059
# the number of recent values to use when smoothing
# the position and rotation of the robot
historyLength = 3

[controls]
# key to reload the configuration at run time. 
#Not all changes will take effect
loadConfig = l
\end{lstlisting}